\documentclass{article}
\usepackage{amsmath}
\usepackage[round]{natbib}
\usepackage{url}
\usepackage{hyperref}

\title{Valuation of Commodity Storage}
\author{Jake C. Fowler}
\date{January 2020}

\begin{document}
\newcommand{\+}[1]{\ensuremath{\mathbf{#1}}}

\maketitle

\section{Introduction}

Define $v_i$ as the decision volume of commodity injected or withdrawn from storage at time
$t_i$. To clarify, positive value of $v_i$ denotes injection into storage, increasing the
inventory, where as a negative value denotes the volume withdrawn. The admissible values
of $v_i$ are restricted by the minimum and maximum inject/withdraw rate functions
$v_{min}$ and $v_{max}$.

\begin{equation}
    v_i \in [v_{min}(t_i, V_i), v_{max}(t_i, V_i)]
\end{equation}

Note that $v_{min}$ and $v_{max}$ are both functions of time and $V_i$, which represents the
inventory in storage at time $t_i$. Inventory-varying injection and withdrawal rates are commonly
seen in natural gas storage, where higher storage cavern pressue results from higher inventory,
which results in a higher maximum withdrawal rate, and lower maximum withdrawal rate.

\bigskip

The inventory $V_i$ can be defined as the sum of all decision volumes $v_i$

\begin{equation}
    V_i = \sum{v_i}
\end{equation}

$V_i$ is itself constrained to be within $V_{min}$ and $V_{max}$, the minimum and maximum
inventory functions.

\begin{equation}
    V_i \in [V_{min}(t_i), V_{max}(t_i)]
\end{equation}

Commonly $V_{min}$ will evaluate to zero for all time periods, but examples
where non-zero minimum inventory is needed include when regulations require a minimum
level of inventory is held, as is seen for natural gas storage in some European countries.
Two possible reasons why $V_{max}$ need to be functions of time are:
\begin{itemize}
    \item Storage could be leased for consecutive time periods, but for different notional
    volumes.
    \item The terms of leased storage commonly stipulate that the storage must be empty
    at the time that the leased capacity ends.
\end{itemize}


Mathematical constraints.

- Set of decision times.

- Decision volume (inject/withdraw) for each time.

- Set of admissible decision volumes which adhere to constraints.

- Definition of inventory. Sum of decision volume, minus volume consumed on,
 decision, minus inventory loss.


\bigskip

Cash flows.

- NPV from inject/withdraw, equals decision volume, adjusted for lost volume,
times cmdty forward price, discounted.

- NPV from inject/withdraw cost.

- NPV from inventory cost.

- Terminal cmdty NPV function.

\bigskip

Valuation Theory.

- Optimal value as risk-neutral expectation for NPVs.

- Bellman equation.

\bigskip

Valuation In Practice.

- Bang-bang assumption: reduce the decision volume set.

- Discretise the set of inventories at which the Bellman equation is evaluated.

- Problem then boils down to calculating the expectation.

\end{document}