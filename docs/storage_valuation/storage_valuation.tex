\documentclass{article}
\usepackage{amsmath}
\usepackage{amssymb} % for "\mathbb" macro
\usepackage[round]{natbib}
\usepackage{url}
\usepackage{hyperref}

\title{Valuation of Commodity Storage}
\author{Jake C. Fowler}
\date{January 2020}

\begin{document}
\newcommand{\+}[1]{\ensuremath{\mathbf{#1}}}

\maketitle

\section{Introduction}
Define $v_i$ as the decision volume of commodity injected or withdrawn from storage at time
$t_i$. To clarify, positive value of $v_i$ denotes injection into storage, increasing the
inventory, where as a negative value denotes the volume withdrawn. The admissible values
of $v_i$ are restricted by the minimum and maximum inject/withdraw rate functions
$v_{min}$ and $v_{max}$.

\begin{equation}
    v_i \in [v_{min}(t_i, V_i), v_{max}(t_i, V_i)]
\end{equation}

Note that $v_{min}$ and $v_{max}$ are both functions of time and $V_i$, which represents the
inventory in storage at time $t_i$. Inventory-varying injection and withdrawal rates are commonly
seen in natural gas storage, where higher storage cavern pressue results from higher inventory,
which results in a higher maximum withdrawal rate, and lower maximum withdrawal rate.

\bigskip

The inventory $V_i$ can be defined recursively as:

\begin{equation}
    V_i = V_{i-1} + v_{i-1} - L(t_{i-1}, V_{i-1}) 
\end{equation}

Where $L(t_{i-1}, V_{i-1})$ is the inventory loss, as a function of time and inventory, 
evaluated for the previous time period. An alternative representation of inventory is 
as a summation:

\begin{equation}
    V_i = V_{start} + \sum\limits^{i-1}_{j=0} {(v_{j} - L(t_j, V_j))}
\end{equation}

$V_{start}$ is the inventory at the inception. This is necessary in the case where storage
capacity is purchased or leased with some amount of commodity inventory in place.

\bigskip

$V_i$ is itself constrained to be within $V_{min}$ and $V_{max}$, the minimum and maximum
inventory functions.

\begin{equation}
    V_i \in [V_{min}(t_i), V_{max}(t_i)]
\end{equation}

Commonly $V_{min}$ will evaluate to zero for all time periods, but examples
where non-zero minimum inventory is needed include when regulations require a minimum
level of inventory is held, as is seen for natural gas storage in some European countries.
Two possible reasons why $V_{max}$ need to be functions of time are:
\begin{itemize}
    \item Storage could be leased for consecutive time periods, but for different notional
    volumes.
    \item The terms of leased storage commonly stipulate that the storage must be empty
    at the time that the leased capacity ends.
\end{itemize}


Mathematical constraints.

- Set of decision times.

- Decision volume (inject/withdraw) for each time.

- Set of admissible decision volumes which adhere to constraints.


\section{Cash Flows}
Define $p(v_i, t_i)$ as the net present (discounted) cash flows resulting from decision
volume $v_i$ at time $t_i$.

\begin{equation}
    p(v_i, t_i, V_i) = (\mu(t_i, V_i, v_i) + v_i) s_i d(c(t_i)) - \pi(t_i, V_i, v_i)
\end{equation}

Where:
\begin{itemize}
    \item $s_i$ is the spot commodity price at time $t_i$.
    \item $c(t_i)$ is the commodity settlement time function which maps from the 
    time that commodity was delivered to the time that payment is made. In European
    energy markets this is usually a formulaic date in the next month.
    \item $d$ is the discount factor function. Encorporating current market risk-free
    interest rates, this function maps from a time of a cash flow to the present value
    of one unit of money. Any cash flow at future time $t$ can be multiplied by $d(t)$
    in order calculate the present value of this cash flow.
    \item $\mu(t_i, V_i, v_i)$ is the volume of commodity consumed (not added to inventory)
    by the storage facility, as a function of time, inventory and decision volume. In practice
    this term is relevant for energy storage where some quantity of the energy commodity
    is consumed in order to power the motors used to get the commodity into or out of storage.
    \item $\pi(t_i, V_i, v_i)$ is the NPV of any other costs which are generated. An example of
    this in practice is the cost of running motors which facilitate injection or withdrawal.
\end{itemize}

The optimal value of the storage at time $t_i$ with inventory $V_i$ can be written as:

\begin{equation}
    \Omega(t_i, V_i) =  \sup_{\+v \in \Phi} \mathbb{E}^Q \biggl[\sum^n_{j=i}p(v_j, t_j, V_j) \biggr]
\end{equation}

Where $\mathbb{E}^Q$ is the expectation operator under the risk-neutral measure, $\+v$ is 
an adapted decision strategy consisting of $\mathcal{F}_{t_i}$ adapted $v_i$ at each time step, 
and $\Phi$ is the set of all feasible decision volume vectors, given the storage 
constraints presented above.

\bigskip
From above %TODO reference the equation
we can see that the storage pricing problem effectively comes down to finding the optimal 
decision strategy. This can be solved using Dynamic Programming. First writing the recursive
Bellman equation.

\begin{equation}
    \label{eq:bellman}
    \Omega(t_i, V_i) = \max_{v_i \in [v_{min}, v_{max}]} \biggl\{ p(v_i, t_i) + \mathbb{E}^Q \biggl[
        \Omega(t_{i+1}, V_i + v_i - L(t_i, V_i)) \biggr] \biggr\}
\end{equation}
% TODO make expectation conditional? Conditional on a vector of Markov variables?
Where the time and inventory dependence of $v_{min}$ and $v_{max}$ has been omitted to lighten notation.
The intuition behind this formula is that at every time step the optimal decision volume $v_i$ 
is the one which maximises the sum of current and discounted future expected cash flows 
conditional upon the decision.
Hence, calculating the value involves recursively solving \ref{eq:bellman}, start at
the end date of the storage facility and moving back in time, at each time step calculating the
optimal decision as a function of the previously calculated values at the next time step.

\subsection{Valuation In Practice}
In order to implement the valuation calculation using \ref{eq:bellman} approximations need
to be made.

\subsubsection{Bang-Bang Exercise Strategy}
One problematic assumption in \ref{eq:bellman} is that the set of all permissible values of
$v_{min}$ is $[v_{min}(t_i, V_i), v_{max}(t_i, V_i)]$. The practical implemenation involves
evaluating the part of \ref{eq:bellman} inside the curly brackets for all permissible values,
picking the value of $v_i$ for which this evaluates to the highest value. However, assuming
$v_{min}(t_i, V_i) < v_{max}(t_i, V_i)$, there are an infinite number of permissible values for
$v_i$ making this impractical to implement. The approximating solution is to assume a 
"Bang-Bang" exercise strategy, this being that the decision is either to inject to storage
at the maximum rate, withdraw at the maximum rate, or do nothing. Expressing this mathematically:

\begin{equation}
    \label{eq:bb-dcsn-set}
    v_i \in \{v_{min}(t_i, V_i), 0, v_{max}(t_i, V_i)\}
\end{equation}

If either $v_{min}(t_i, V_i) > 0$ or $v_{max}(t_i, V_i) < 0$ then the 0 element is removed 
from \ref{eq:bb-dcsn-set} as it is no longer in the real set of permisible values.

\bigskip

Using a "Bang-Bang" exercise strategy is an approximation, hence the decision stategy that
this results in could be suboptimal, giving a lower value for the storage than it's true
value. However, several studies have shown that this assumption is a realistic one, and that
even if the set of permissible values for $v_i$ includes a larger number of elements, the
optimal strategy will be very close to the "Bang-Bang" strategy, and hence the calculated 
storage value will be not far from the true optimal value.
To do: provide references.


- Discretise the set of inventories at which the Bellman equation is evaluated.

- Problem then boils down to calculating the expectation.

\end{document}